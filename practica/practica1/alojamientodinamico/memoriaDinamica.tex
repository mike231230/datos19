\documentclass[10pt,a4paper]{article}
\usepackage[utf8]{inputenc}
\usepackage[activeacute,spanish]{babel}
\usepackage{graphicx}
\author{Miguel Angel Aguirre Olvera}
\title{Practica 1 memoria dinamica}
\usepackage{multicol}


\begin{document}
\maketitle
\begin{multicols}{2}
\begin{center}
\textbf{¿Que es la memoria dinamica?}
\end{center}
Es memoria que se reserva en tiempo de ejecucion. Su principal ventaja frente a la memoria estatica es que su tamaño puede variar durante la ejecucion de un programa. En C el programador es el encargado de liberar esta memoria cuando no la utilice mas. El uso de memoria dinamica es necesario cuando no conocemos el numero de datos o elementos a tratar; sin embargo es algo mas lento ya que el tiempo de ejecucion depende del espacio que se va a usar, en cambio la memoria estatica es mas rapida ya que esta disponible desde que se inicio el programa 

\begin{center}
\textbf{codigo}
\end{center}
  include iostream
 	using namespace std;
 	
	struct persona
		char nombre[25];
		int edad;
		char id[10];
	
	;

 	int main
 	 
 	persona *persona1, *persona2;

 	persona1= new persona ;

 	cout dame el nombre de la persona endl;

 	cin persona1 nombre;

 	cout dame la edad endl;

 	cin persona1 edad;

 	cout dame el numero de identidad endl;

 	cin persona1 id;

	 	persona2= new persona();

 	cout dame el nombre de la persona endl;

 	cin persona2 nombre;

 	cout dame la edad endl;

 	cin persona2 edad;

 	cout dame el numero de identidad endl;

 	cin persona2 id
 	

 	cout persona1 nombre persona1 edad persona1 id endl;

 	

 	cout persona2 nombre persona2 edad persona2 id endl;

 	

	delete persona1;


	delete persona2;
 	
 	
 	
 	return 0;

\end{multicols}

\end{document}